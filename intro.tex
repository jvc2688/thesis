\chapter*{Introduction}\addcontentsline{toc}{chapter}{Introduction}

\newcommand{\pdc}{\project{PDC}}


\section{Self-calibration in wide-field imaging surveys}
As with all physics experiments, a common challenge in astronomical research is relating detector output to the underlying physical quantity of interest---calibration.

Traditionally, astronomical imaging calibration is conducted in an absolute way, in which observations are compared with calibrated standard objects.
Such calibration schemes are not optimal for the following reasons:
Calibration objects and science targets are not observed in the same condition (in terms of detector location, colors, exposure time and so on), while the point-spread function (PSF), flat-fields, and so on depend on specific observational configurations.
To compare with standard objects, relating different telescope configurations can't be avoided, so the calibration performance is also limited by the accuracy of transformations between different systems.

However in many astronomical applications,  the absolute scale of the physical quantity is not very important.
Instead, achieving a consistent relative calibration across the whole dataset is more urgent.
This provides an opportunity to evade/obviate the disadvantages of traditional calibration, by separating the relative and absolute calibration and only using data from the native telescope system and configuration---self-calibration.

\sdss\ self-calibration \citep{uber} separated the relative calibration from the absolute calibration and make use of the repeat observations to constrain the photometric calibrations, which achieved 1\% relative calibrations/spatial error modes compared with the  previous $\sim$ 2\% relative calibration accuracy \citep{sdss2}.

In \chap{gs}, we will extend the similar idea to the space-based survey---The Galaxy Evolution Explorer (\galex, \citealt{galex1}), which will deliver the unique images and star catalog in ultraviolet around the Galactic Plane. 

\section{Data-driven model}
As mentioned before, the process of the calibration is to find a mapping function between instrument output and the underlying physical quantity of interest.
The mapping function (or the model) can be either a physical model or it can be a flexible, effective model that has no direct interpretation in terms of physical parameters---a data-driven model.
In a physical model, this mapping function is usually decomposed into different parts including telescope pointing, PSF, flat-field, distortion map and so on.
The advantage of the physical model is that it's easy to input the prior information and can be interpreted.
Therefore a physical model is expected to perform better because of the additional prior information.
However if the prior knowledge or the intuition about the dominant effects is wrong or incomplete, then a physical model may perform worse.

Instead a data-driven model is harder to incorporate prior information but also prevents loss from wrong heuristic. 
An successful example of this kind of flexible, effective models is the \kepler\ Presearch Data Conditioning (\pdc, \citealt{pdc1}, \citealt{pdc2}, \citealt{pdc3}). 
The \pdc\ builds on the idea that systematic errors have a temporal structure that can be extracted from ancillary quantities.
\pdc\ ``co-trends'' or calibrates the \kepler\ light curves by removing those parts that are explained by the basis light curves generated from a principal components analysis (PCA) of filtered light curves.
That is, it exploits statistical dependences between different time series, regularizing the fit (and avoiding over-fitting) by filtering and restricting the dimensionality (through PCA).
Relying on the \pdc, thousands of the exoplanets candidates have been discovered in \kepler\ (\citealt{kepler}, \citealt{kepler1}).

In \chap{cpm}, we will discuss more about the data-driven model as well as incorporate the underlying causal structure into the model.

\section{Crowded fields photometry}
One of the biggest challenge in astronomical observation is doing photometry in a crowded fields.
Crowded fields photometry is hard because stars in the field are not be isolated.
Therefore common aperture photometry is not applicable in this situation.

A conventional approach for crowded fields photometry is to reduce images into catalogs of point sources and build a comprehensive model for the point-spread function (PSF),  the sky background, and so on (\citealt{psf1}, \citealt{psf2}).
The challenges in this kind of approach include but not limited to imperfect deblending, sources detection and PSF fitting.
The model complexity also scales with the number of the sources in the field (\citealt{crowd1}, \citealt{crowd2}, \citealt{crowd3}).

However in many situations, it is only the variable sources and the differential photometry between image frames are interested.
Therefore the complexities of photometry in a crowded field can be greatly reduced, if the constant sources in the field can be properly removed.

Difference imaging or image subtraction (\citealt{imagesub1}, \citealt{alard}, \citealt{varyingkernel}) is a method developed for detecting and doing differential photometry for variable objects in crowded fields, in which the difference is measured between two images that are both positionally and photometrically matched.
The common workflow for difference imaging is:
\begin{enumerate}
\item
Create a reference image by either stacking image frames or selecting the frame with the best seeing.
\item
Astrometrically register each frame to the reference frame.
\item
Match the seeing between each image and the reference frame by fitting a convolution kernel that accounts for the differences between the point spread functions(PSFs).
\item
Match the mean throughput or photometric calibration of the two frames and subtract to get a difference image.
\end{enumerate}
Difference imaging has been successfully applied for the detection of variable sources in microlensing \citep{macho, ogle} and supernova \citep{sdss} surveys.

Different approaches of difference imaging (\citealt{imagesub1}, \citealt{alard}, \citealt{varyingkernel}, \citealt{bramich}) differ in the way how the convolution kernel is estimated.
However all these difference imaging methods only utilize the spatial constrain between nearby pixels of two image frames.
There are situations that data are taken very differently like \ktwo\ (\citealt{k2}), in which thousands of images of the same field are taken in different time stamps.
These kind of data could provide additional structure and constrain in time-domain.
In \chap{cdi}, we proposed a totally different method that do not make a reference image and make a model of how pixels predict one another, using the full time history of the data.

\section{Outline and contributions}
\chap{cpm} has been refereed and published in the astronomical literature.
\Chap{cdi} has been submitted to \emph{The Astrophysical Journal}.
\Chap{gs} is in preparation for submission.
All of these \chapname s were co-authored with collaborators but the majority of the work and writing in each \chapname\ is mine.
Here, I describe my specific contributions to each \chapname:
\begin{enumerate}

{\item For \chap{gs}  I developed the project idea in collaboration with David Hogg, David Schiminovich and Steven Mohammed.
I implemented the project with contributions from Steven Mohammed.
The project utilized software written by Chase Million.
I wrote the paper with additions by David Hogg and David Schiminovich.}

{\item The fundamental ideas underlying \chap{cpm} were developed through discussions with David Hogg, Daniel Foreman-Mackey and Bernhard Sch\"olkopf.
I then implemented the project and wrote the paper with Sections contributed by David Hogg and additions by Daniel Foreman-Mackey and Bernhard Sch\"olkopf.}

{\item For \chap{cdi} I developed the idea for the project in collaboration with David Hogg, Daniel Foreman-Mackey, and Bernhard Sch\"olkopf.
I implemented the project and wrote the paper with additions by David Hogg, Daniel Foreman-Mackey and Bernhard Sch\"olkopf.
}
\end{enumerate}
